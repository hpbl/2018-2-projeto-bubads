%% 
%% Copyright 2007-2018 Elsevier Ltd
%% 
%% This file is part of the 'Elsarticle Bundle'.
%% ---------------------------------------------
%% 
%% It may be distributed under the conditions of the LaTeX Project Public
%% License, either version 1.2 of this license or (at your option) any
%% later version.  The latest version of this license is in
%%    http://www.latex-project.org/lppl.txt
%% and version 1.2 or later is part of all distributions of LaTeX
%% version 1999/12/01 or later.
%% 
%% The list of all files belonging to the 'Elsarticle Bundle' is
%% given in the file `manifest.txt'.
%% 
%% Template article for Elsevier's document class `elsarticle'
%% with harvard style bibliographic references

\documentclass[preprint,12pt,authoryear]{elsarticle}

%% Use the option review to obtain double line spacing
%% \documentclass[authoryear,preprint,review,12pt]{elsarticle}

%% Use the options 1p,twocolumn; 3p; 3p,twocolumn; 5p; or 5p,twocolumn
%% for a journal layout:
%% \documentclass[final,1p,times,authoryear]{elsarticle}
%% \documentclass[final,1p,times,twocolumn,authoryear]{elsarticle}
%% \documentclass[final,3p,times,authoryear]{elsarticle}
%% \documentclass[final,3p,times,twocolumn,authoryear]{elsarticle}
%% \documentclass[final,5p,times,authoryear]{elsarticle}
%% \documentclass[final,5p,times,twocolumn,authoryear]{elsarticle}

%% For including figures, graphicx.sty has been loaded in
%% elsarticle.cls. If you prefer to use the old commands
%% please give \usepackage{epsfig}

%% The amssymb package provides various useful mathematical symbols
\usepackage{amssymb}
\usepackage[utf8]{inputenc}
%% The amsthm package provides extended theorem environments
%% \usepackage{amsthm}

%% The lineno packages adds line numbers. Start line numbering with
%% \begin{linenumbers}, end it with \end{linenumbers}. Or switch it on
%% for the whole article with \linenumbers.
%% \usepackage{lineno}

\journal{if105 - Data Science - UFPE}

\begin{document}

\begin{frontmatter}

%% Title, authors and addresses

%% use the tnoteref command within \title for footnotes;
%% use the tnotetext command for theassociated footnote;
%% use the fnref command within \author or \address for footnotes;
%% use the fntext command for theassociated footnote;
%% use the corref command within \author for corresponding author footnotes;
%% use the cortext command for theassociated footnote;
%% use the ead command for the email address,
%% and the form \ead[url] for the home page:
%% \title{Title\tnoteref{label1}}
%% \tnotetext[label1]{}
%% \author{Name\corref{cor1}\fnref{label2}}
%% \ead{email address}
%% \ead[url]{home page}
%% \fntext[label2]{}
%% \cortext[cor1]{}
%% \address{Address\fnref{label3}}
%% \fntext[label3]{}

\title{Análise de READMEs de projetos open-source}

%% use optional labels to link authors explicitly to addresses:
%% \author[label1,label2]{}
%% \address[label1]{}
%% \address[label2]{}

\author{Bruno Melo}
\ead{bhlvm@cin.ufpe.br}

\author{Hilton Leite}
\ead{hpbl@cin.ufpe.br}

% \address{}

\begin{abstract}
Repositórios open-source fazem uso de arquivos README escritos em markdown, para introduzir os conceitos dos mesmos, porém existem poucas ferramentas que analisam a efetividade e os padrões de escrita destes arquivos. Nesta proposta detalhamos motivação, forma de coleta dos dados, e possíveis análises a serem realizadas.
\end{abstract}

\begin{keyword}
%% keywords here, in the form: keyword \sep keyword
Ciência dos dados \sep README \sep open-source \sep Developer experience
%% PACS codes here, in the form: \PACS code \sep code

%% MSC codes here, in the form: \MSC code \sep code
%% or \MSC[2008] code \sep code (2000 is the default)

\end{keyword}

\end{frontmatter}

%% \linenumbers

%% main text
\section{Introdução}
\label{}
O arquivo README.md de um projeto open-source é de grande importância para o futuro do repositório, pois ele é o ponto de entrada para possíveis novos usuários e/ou mantenedores. Nele geralmente se descreve os detalhes de uso e contribuição do projeto. Um README mal redigido pode ser um fator que faz o usuário desisitir de usar o projeto.


\section{Coleta dos Dados}
Para a análise usaremos apenas repositórios hospedados no GitHub. Como o escopo é reduzido a projetos open-source, os arquivos README necessários são de fácil acesso, necessitando apenas acessar a URL do repositório. Dados complementáres sobre issues criados relativos ao README, ou quaisquer outras informações sobre o repositórios podem ser adquiridas via a API do GitHub.

%% The Appendices part is started with the command \appendix;
%% appendix sections are then done as normal sections
%% \appendix

\section{Análise dos Dados}
Faremos uma análise exploratória dos arquivos coletados, de forma a tentar descobrir padrões de escrita, tipo de linguagem utilizada, quais sessões constam, e quantos issues relacionados ao README foram abertos.

Buscaremos tentrar evidenciar quais são as práticas comuns na produção dos README, e quais os problemas mais comumns encontrados nos repositórios, com auxílio de visualizações comom word-clouds. De forma com que possamos identificar padrões e anti-padrões e disponibilizá-los para utilidade pública dos mantenedores de repositórios.

%% If you have bibdatabase file and want bibtex to generate the
%% bibitems, please use
%%
%%  \bibliographystyle{elsarticle-harv} 
%%  \bibliography{<your bibdatabase>}

%% else use the following coding to input the bibitems directly in the
%% TeX file.

\begin{thebibliography}{00}

%% \bibitem[Author(year)]{label}
%% Text of bibliographic item

\bibitem[ ()]{}

\end{thebibliography}
\end{document}

\endinput
%%
%% End of file `elsarticle-template-harv.tex'.
